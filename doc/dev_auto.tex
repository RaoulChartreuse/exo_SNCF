\section{Automatisation de la machine de dev}
Le principe de cette automatisation est d'utiliser des conteurs docker qui vont être réutilisé par kubernetes.

\subsection{Ecriture des dockerfile}
Les dockerfiles sont simples.
Pour le serveur appache ont utilise seulement un serveur php-apache en verision 7.2. La seul modification est l'ajout du driver mysqli et du code php.

\lstinputlisting[]{../src/dockerfile_web}

Pour la base de donnée on part de la denière version du conteneur mariadb, et on copie les scripts pour initialiser la base de donnée.

\lstinputlisting[]{../src/dockerfile_db}

\subsection{Docker-compose}
Les conteneurs sont déployés localement grace à docker-compose.

\lstinputlisting[]{../docker-compose.yml}

\subsection{Deploiment continue avec Jenkins}
Sur le serveur d'integration on a un serveur Jenkins qui nous permets de metre en place un job qui surveille git.
Pour chaque mise à jour de la branch dev, mets à jours de docker-compose de la machine de dev.
Ce qui permets à la machine de dev d'être toujours à jours.

\lstinputlisting[]{../src/jenkins/dev.sh}
