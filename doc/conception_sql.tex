\section{Conception SQL}
\subsection{Trajet}
\begin{itemize}
\item codeTrajet : entier autoincrement et clée primaire
\item gareDepart : entier clé primaire de l'objet Gare
\item gareArrivee : entier clé primaire de l'objet Gare
\item conducteur : entier clé primaire de l'objet Conducteur
\item Rame : entier clé primaire de l'objet Rame
\item date : type DATE
\item duree : type TIME
\end{itemize}

\subsection{Gare}
\begin{itemize}
\item codeGare : entier autoincrement et clée primaire
\item nom : VARCHAR(50)
\item Ville : VARCHAR(50)
\item addresse : VARCHAR(250)
  item Nquai : entier
\end{itemize}

\subsection{Rame}
\begin{itemize}
\item codeRame : entier autoincrement et clée primaire
\item type : VARCHAR(10) (code interne potentiellement à une autre table de description)
\item nPremiere : entier
\item nSeconde : entier
\end{itemize}


\subsection{Conducteur}
\begin{itemize}
\item codeConducteur : entier autoincrement et clée primaire
\item nom : VARCHAR(50)
\item prenom : VARCHAR(50)
\item secu : VARCHAR(13) 
\end{itemize}



%%% Local Variables:
%%% mode: latex
%%% TeX-master: t
%%% End:
