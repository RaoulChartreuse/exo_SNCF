*\section{Automatisation du serveur de prod}
Le principe de ce serveur est d'utiliser les conteneurs du serveur dev un fois ceux-ci validé.
Et de les utiliser avec kubernetes.

\subsection{Installation de minikube et kubectl avec ansible}
L'installation de minikube et kubectl est obtenue avec  un playbook ansible qui est lancé une seul fois par machine.

\lstinputlisting[language=bash]{../src/bash/install_minikube.sh}

\subsection{Configuration de kubernetes}
La configuration de kubernetes est obtenue par l'écriture des fichier de configuration décrivant les volumes, les déploiements et les services.

\subsubsection{Volume}

\lstinputlisting[]{../src/k8s/volume.yml}

\subsubsection{Deployment}
La gestion des tags des images utilisées sera décrite plus tard \ref{prod_jenkins}.
\lstinputlisting[]{../src/k8s/deploy.yml}

\subsubsection{Service}
Le pod php-apache peut contacter la base de donnée en utilisant la variable servername=db,
on conserve ainsi le même ficher de configuration entre le serveur dev et le serveur prod.

\lstinputlisting[language=php]{../src/php/configDb.php}

\lstinputlisting[]{../src/k8s/svc.yml}

\subsection{Déploiement continue avec Jenkins}
\label{prod_jenkins}
Le déploiement sera géré par Jenkins.
Il sera déclenché par une surveillance de la branche master qui déclenchera les actions suivantes~:
\begin{itemize}
\item Construction des dernières versions des images docker
\item Envoie des images sur dockerHub avec un tag unique
\item Sur le serveur de prod~: mise à jours du cluster kurbernetes pour utiliser les nouvelles images
\end{itemize}

On ces 2 scripts exécutés par Jenkins respectivement sur le dev et le prod~:

\lstinputlisting[language=bash]{../src/jenkins/master_dev.sh}

\lstinputlisting[language=bash]{../src/jenkins/master_prod.sh}

\subsubsection{Gestion des tags}
Pour la gestion des tag, nous avons décidé de ne pas les gérer manuellement.
Ils se trouvent dans plusieurs fichiers différent, leurs gestion manuelle apportent un risque de défaillance inutiles.

Nous utiliseront le hash de du dernier commit git,  ce qui a l'avantage d’être unique.

Pour modifier les tags utiliser dans le fichier deploy.yml du Kubernetes,
nous utiliseront un outil kustomize qui permets d'appliquer un patch au fichier de configuration.

\lstinputlisting[]{../src/k8s/kustomization.yml}

\subsubsection{Script sur le serveur d'intègration}
Ce script pousse les images à jour sur DockerHub.
L'authentification ce fait avec un mots de passe sur la machine.
Une authentification par token serai préférable.

\lstinputlisting[language=bash]{../src/bash/on_update.sh}

\subsubsection{Script sur le serveur prod}
Ce script récupère les images et applique le patch kustomize pour demander à Kubernetes de changer de version des images.

\lstinputlisting[language=bash]{../src/bash/update_prod.sh}

