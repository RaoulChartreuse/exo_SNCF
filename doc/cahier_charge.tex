\section{Cahier des charges}
Nous souhaitons réaliser un site web pour afficher la liste des TGV disponible
classés par ordre croissant de date de départ.

Un TGV est identifié par un numéro (par exemple 6440) et possède une ville de
départ et une ville d’arrivée, un horaire de départ et une durée (et par conséquent
une date d’arrivée).

Ce projet devra être réalisé avec la technologie php/mysql. Le serveur web devra
proposer $php >5.6$ et la base de données devra être idéalement a la version mysql
5.7.

Nous nous situons plutôt dans le contexte des devops ou nous devons proposer la
plateforme suivante :

\subsection{Sur poste windows :}
installer easyphp ou wamp ainsi que visual studio code (ou
tout IDE qui vous convient). A chaque fois ou on juge que notre travail est “stable”,
nous devons faire un push sur la branche git dev de notre repository git.

Notre pipeline est composé de deux stages : dev et prod

\subsection{Dans le stage dev :}
nous nous limitons à un serveur linux contenant un httpd/php
et mysql. Ce serveur doit toujours proposer toujours la version la plus à jours sur la
branche dev.

\subsection{Dans le stage prod :}
\begin{itemize}
  \item Dans ce stage, nous devons proposer une architecture hautement disponible
qui doit être composée d’un minimum de deux clones des composants
informatiques pour assurer la tolérance aux pannes
  \item Dans ce stage, nous devons déployer la dernière version de la branche
    master de git sur la plateforme du stage prod.
  \item Prévoir une solution pour que votre infrastructure soit extensible, i.e, dans le
cas ou on se rend compte qu’il y aurait une surchage des serveurs, nous devons mettre à échelle (scaling) notre architecture en un temps
relativement court.
\end{itemize}

\subsection{Votre mission :}
\begin{itemize}
\item Concevoir le diagramme UML de l‘application
\item Concevoir et implémenter la base de données
\item Insérer un jeu de données des tests
\item exporter le schéma (et uniquement le schéma) dans un fichier sql
\item développer un fichier html statique qui affiche un message de bienvenu
\item Créer un repository git et faites un “push” initial des fichiers html et sql
\item La suite des tâches dépendent de votre conception de la solution conçue
\end{itemize}
%%% Local Variables:
%%% mode: latex
%%% TeX-master: t
%%% End:
