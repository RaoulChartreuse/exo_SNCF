\section{Conclusion}
Nous avons réussi à obtenir le fonctionnement de l'application tant sur le serveur de dev que sur le serveur de prod.
Nous avons réussi à obtenir une automatisation de l'intégration que ce soit sur un serveur de développement que sur un serveur de production.
Cela en utilisant la technologie docker-compose qui à l'avantage d'être simples à mettre en oeuvre,
et la technologie Kubernetes qui est plus adaptée à un environnements de production par exemple avec la réplication des pods.

Il y a encore de nombreuse amélioration possible.
Les voici en vrac sans ordres de priorités~:

\subsection{Jenkins}
Les scripts sont écrit à l'aide de l'interface graphique de Jenkins.
Le passage aux scripts textes aurait l'avantage de permettre un versionnent de ceux-ci et
une mise en place facilitée sur un autre serveur Jenkins.

\subsection{Kubernetes}
Les mises à jours sont obtenue à partir d'un script qui s'exécute sur le
serveur de prod.
Il serait préférable d'utiliser l'api Kubernetes depuis le serveur d’intégration.
Cela aurait aussi l'avantage de permettre un monitoring et une modification de la configuration en fonction de la charge.

\subsection{DockerHub}
L'authentification avec DockerHub ce fait par un mot de passe.
On pourrait créer un token.

Le problème principale et le fait que les images sont public.
Il faudrait souscrire à DockerHub pour pouvoir conserver plusieurs images privées.
Comme nous n'avons pas besoin de diffuser ses images,
il est plus raisonnable de mettre en place un serveurs registry pour docker.


\subsection{Le code php}
Le code php utiliser ici est une simple ébauche pour vérifier le fonctionnement de l'application php et la connexion et la modification de la base de donné.

L'application doit être écrite maintenant.

\subsection{La base de donnée}
La sauvegarde de la base de donnée n'est pas prise en charge actuellement.
La sauvegarde et l’automatisation sont des fonctionnalités à implémenter dans une possible suite au projet.
